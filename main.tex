\documentclass[useAMS,usenatbib,usegraphicx]{mn2e}

%% Language and font encodings
\usepackage[english]{babel}
\usepackage[utf8x]{inputenc}
\usepackage[T1]{fontenc}

%% Useful packages
\usepackage{natbib}
\usepackage{amsmath}
\usepackage{amssymb}
\usepackage{graphicx}
\usepackage[colorlinks=true, allcolors=blue]{hyperref}
\usepackage{aas_macros}
\usepackage{bm}
\usepackage{algorithm, algorithmic}
\usepackage{color}

\newcommand{\refcomment}[1]{{\color{red} #1}}

\title[Reduced Order Modelling for Continuous GWs]{Reduced Order Modelling in searches for continuous
gravitational waves - II. reduced order quadrature for fast likelihood evaluatoin}
\author[M.~Pitkin]{M.~Pitkin$^1$\thanks{matthew.pitkin@glasgow.ac.uk} \\
$^1$SUPA, School of Physics and Astronomy, University of Glasgow,
University Avenue, Glasgow, G12 8QQ, UK \\

\pagerange{\pageref{firstpage}--\pageref{lastpage}} \pubyear{2018}

\begin{document}

\label{firstpage}

\maketitle

\begin{abstract}
Describe using ROQ for speeding up likelihood evaluation for gravitational wave known pulsar searches.

\end{abstract}

\begin{keywords}
gravitational waves, pulsars: general, methods: data analysis
\end{keywords}

\section{Introduction}\label{sec:intro}

An intro...

\section*{Acknowledgements}

This work has benefited greatly from discussions with Rory Smith, and from many
discussions with members of the LIGO Scientific Collaboration and Virgo Collaboration,
in particular members of the continuous waves working group. The analysis has
relied on the {\sc greedycpp} software \citep{greedycpp} and LALSuite \citep{LALSuite}. The analysis
has also been greatly aided by the use of IPython \citep{PER-GRA:2007}, jupyter notebooks 
\citep{kluyver2016jupyter} and Cython \citep{behnel2010cython}, and all plots have been produced 
using Matplotlib \citep{Hunter:2007,michael_droettboom_2017_248351}. 
MP is funded by the STFC under 
grant number ST/N005422/1.  This paper carries LIGO Document Number LIGO-P18XXXXX.

\bibliographystyle{mn2e}
\bibliography{main}

\label{lastpage}

\end{document}

